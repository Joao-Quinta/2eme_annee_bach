\documentclass[12pt]
{report}

\usepackage[utf8x]{inputenc}
\usepackage[T1]{fontenc}
\usepackage[french]{babel}
\usepackage{fancyhdr}
\usepackage{xcolor}

\definecolor{blue}{rgb}{0,0,1}
\definecolor{red}{rgb}{1,0,0}
\definecolor{green}{rgb}{1,0,1}
\newcommand\tab[1][1cm]{\hspace*{#1}}

\begin{document}
% Entete
\pagestyle{fancy}
\lhead{Joao Quinta}
\rhead{16.03.2020}
\chead{\textbf{Semantique}}

\begin{center}
	\section*{\textbf{TP1}}
\end{center}
\bigskip
\paragraph*{\textbf{Exercice 1:}}
\subparagraph*{\textbf{.1)}}

\begin{flushleft}
\tab Voici comment sont organisées les salles :\tab\tab 
\begin{tabular}{|c|c|c|c|}
\hline
1 & 2 & 3 & 4\\
\hline
5 & 6 & 7 & 8\\
\hline
9 & 10 & 11 & 12\\
\hline
13 & 14 & 15 & 16\\
\hline
\end{tabular}
\end{flushleft}
\tab\textbf{porte(2,1)} $\rightarrow$ veut dire qu'il existe une porte qui va de la salle 2 à la salle 1.
\\ \tab Chaque porte est défini comme ça au début du fichier .pl. \\ 

\subparagraph*{\textbf{.2)}} Dans la salle (1) (haut à gauche), il existe une sortie, qui est défini comme \textbf{sortie(1)} dans le fichier .pl, toute sortie, entree et minotaure seront défini ainsi.  


\bigskip
\paragraph*{\textbf{Exercice 2:}}
\subparagraph*{\textbf{.1)}} Règle \textbf{chemin(De,Vers)} return true si le chemin depuis De à Vers existe, false sinon.
\\ \tab \textbf{chemin(X,X).} règle de base, qui dit qu'il existe un chemin de X à X \textbf{(De = Vers)}.
\\ \tab Si \textbf{De != Vers}, alors on "demande" à prolog s'il existe une porte sortante de \textbf{De}, avec la commande \textbf{porte(De, X)}, cette commande retourne dans \textbf{X} toute salle accessible depuis \textbf{De}. Ensuite nous appelons récursivement en remplacent \textbf{De} par le \textbf{X} trouvé, la logique, est de plutôt que calculer/chercher, un chemin de \textbf{De} à \textbf{Vers}, on cherche s'il existe une porte de \textbf{De} à \textbf{X}, et ensuite un chemin de \textbf{X} à \textbf{Vers}.


\subparagraph*{\textbf{.2)}} Règle \textbf{itineraire(De,Vers,Pieces)}, retourne dans \textbf{Pieces}  une liste des salles à parcourir pour aller de \textbf{De} à \textbf{Vers} (si le chemin existe).
\\ \tab La logique appliqué dans cette règle est la même que dans \textbf{chemin(De,Vers)}. 
\\ \tab \textbf{chemin(X,X,[X]).} règle de base, quand \textbf{De = Vers}, alors on retourne sous forme de liste, l'élément \textbf{De}.
\\
\\ \tab On initialise le backtracking pour construire la liste, vu que la valeur \textbf{Vers} est déjà dans la liste, du à la règle de base. Du coup j'ai fait un \textbf{(if -> then; else)} qui vérifie qu'on est pas dans l'étape ou \textbf{X = Vers}, si: \\ \tab \textbf{X != Vers} $\rightarrow$ on concatène la liste précédente avec \textbf{[X]}. \\ \tab \textbf{X = Vers} alors on concatène une liste vide à la liste précédente, car cette liste précédente est exactement \textbf{[X]}.
\\ 
\begin{center}
\textbf{{\large !!!}}
\end{center}
\tab Par exemple, pour \textbf{itineraire(3,1,Pieces) $\rightarrow$ Pieces = [3, 2,1]}, veut dire qu'en partant de la salle 3, on prend la porte vers la salle 2, et ensuite pour la salle 1.
\begin{center}
\textbf{{\large !!!}}
\end{center}

\paragraph*{\textbf{Exercice 3:}}
\subparagraph*{\textbf{.1)}} La règle \textbf{batterie(Pieces,Batterie,Reste)} retourne dans \textbf{Reste} la batterie restante après avoir parcourue les salles dans \textbf{Pieces}. Cette règle est très simple, elle calcule la taille de la liste \textbf{Pieces} et soustrait cette valeur à \textbf{Batterie}. Par contre, si il n'y a pas assez de batterie pour parcourir toutes les salles, la règle retourne \textbf{false}.
\\ \tab Mon code retourne un résultat attendu lors que je tente la règle \\ \textbf{test$\_$batterie(De,Vers,Batterie,Reste)}.


\subparagraph*{\textbf{.2)}} La règle \textbf{chemin$\_$batterie(De,Vers,Batterie,Pieces,Reste)} retourne dans \textbf{Pieces} un chemin qui va de \textbf{De} à \textbf{Vers}, elle utilise donc la règle \textbf{itineraire(De,Vers,Pieces)} définie avant dans le TP, ensuite avec la liste \textbf{Pieces} elle utilise la règle \textbf{batterie(Pieces,Batterie,Reste)} pour savoir si ce chemin \textbf{Pieces} est réalisable avec la \textbf{Batterie} à disposition. La règle \textbf{batterie(Pieces,Batterie,Reste)} retourne déjà \textbf{false} si le chemin n'est pas réalisable, donc je n'ai pas besoin de faire d'autres changements.

\subparagraph*{\textbf{.3)}}La règle \textbf{chemin$\_$reussite(Batterie,Pieces)}, cherche les sorties, entrees et le minotaure avec les commandes \textbf{entree(X), sortie(Y) et minotaure(Z)} respectivement, ensite on utilise la règle \\ \textbf{chemin$\_$batterie(X,Y,Batterie,Pieces,Reste)}, qui nous rend tous les chemins possibles de \textbf{X} à \textbf{Y} avec la \textbf{Batterie} disponible, ensuite avec la règle prédéfinie \textbf{member(Z,Pieces)}, on trie les chemins qui contiennent l'enplacement du Minotaure(\textbf{Z}).


\paragraph*{\textbf{Exercice 4:}}
\subparagraph*{\textbf{.1.1)}} La règle \textbf{reussite$\_$complete(Batterie,Pieces)}, cherche les sorties, entrees et le minotaure avec les commandes \textbf{entree(X), sortie(Y) et minotaure(Z)} respectivement, ensuite un appel à la règle \textbf{itineraire(X,Y,Pieces)} qui retourne un chemin dans \textbf{Pieces} qui va de \textbf{X} à \textbf{Y}, ensuite on vérifie si dans ce chemin il y a la salle \textbf{Z} (minotaure):
\\ \tab Si \textbf{oui} $\rightarrow$ on utilise la règle \textbf{batterie(Pieces,Batterie,Reste)}, sauf que ici la valeur dans \textbf{Batterie} vaut \textbf{Batterie = Batterie - 5 (énergie lumière) - 2 (énergie tweet)}.
\\ \tab Si \textbf{non} $\rightarrow$ on retourne \textbf{false}, car le chemin dans \textbf{Pieces} ne permet pas de battre le minotaure.
\\ \\ \tab Oui, il est possible de vaincre le Minotaure, tweeter et sortir du labyrinthe avec une batterie de 15. (J'ai aussi pris en compte 5 de énergie pour la lumière).
\\ \\ \tab Légende: \textcolor{red}{salles visités} | \textcolor{blue}{sortie/entree} | \textcolor{green}{Minotaure}
\\ \\ \tab Pour une \textbf{Batterie = 14}, les chemins suivants sont possibles:

\begin{center}

\begin{tabular}{|ccccccc|}
\hline
1 & & 2 & & 3 & & 4 \\
&&&&&& \\
5 & & \textcolor{red}{6} &$\rightarrow$& \textcolor{green}{7} & & 8 \\
&&$\uparrow$&&$\downarrow$&& \\
\textcolor{red}{9} &$\rightarrow$& \textcolor{red}{10} & & \textcolor{blue}{11} & & 12 \\
$\uparrow$&&&&&& \\
\textcolor{blue}{13} & & 14 & & 15 & & 16 \\
\hline
\end{tabular}
\end{center}

\begin{center}
\begin{tabular}{|ccccccc|}
\hline
\textcolor{blue}{1} &$\leftarrow$ & \textcolor{red}{2} &$\leftarrow$ & \textcolor{red}{3} & & 4 \\
&&&&$\uparrow$&& \\
5 & & \textcolor{red}{6} &$\rightarrow$& \textcolor{green}{7} & & 8 \\
&&$\uparrow$&&&& \\
\textcolor{red}{9} &$\rightarrow$& \textcolor{red}{10} & & 11 & & 12 \\
$\uparrow$&&&&&& \\
\textcolor{blue}{13} & & 14 & & 15 & & 16 \\
\hline
\end{tabular}
\end{center}

\begin{center}
\begin{tabular}{|ccccccc|}
\hline
1 & & 2 & & 3 &  & 4 \\
&&&&&& \\
5 & & \textcolor{red}{6} &$\rightarrow$& \textcolor{green}{7} &  & 8 \\
&&$\uparrow$&&$\downarrow$&& \\
9 & & \textcolor{red}{10} & & \textcolor{blue}{11} &  & 12 \\
&&$\uparrow$&&&& \\
13 & & \textcolor{red}{14} &$\leftarrow$& \textcolor{red}{15} &$\leftarrow$& \textcolor{blue}{16} \\
\hline
\end{tabular}
\end{center}


\subparagraph*{}
\tab Pour une \textbf{Batterie = 15}, les chemins suivants sont possibles (les chemins \textbf{Batterie = 14} sont inclus):\\

\begin{center}
\begin{tabular}{|ccccccc|}
\hline
\textcolor{blue}{1} &$\leftarrow$ & \textcolor{red}{2} &$\leftarrow$ & \textcolor{red}{3} & & 4 \\
&&&&$\uparrow$&& \\
5 & & \textcolor{red}{6} &$\rightarrow$& \textcolor{green}{7} &  & 8 \\
&&$\uparrow$&&&& \\
9 & & \textcolor{red}{10} & & 11 &  & 12 \\
&&$\uparrow$&&&& \\
13 & & \textcolor{red}{14} &$\leftarrow$& \textcolor{red}{15} &$\leftarrow$& \textcolor{blue}{16} \\
\hline
\end{tabular}
\end{center}





\subparagraph*{}
\tab Pour une \textbf{Batterie = 16}, les chemins suivants sont possibles (les chemins \textbf{Batterie = 14 et 15} sont inclus):\\

\begin{center}
\begin{tabular}{|ccccccc|}
\hline
\textcolor{blue}{1} &$\leftarrow$ & \textcolor{red}{2} &$\leftarrow$ & \textcolor{red}{3} &$\leftarrow$& \textcolor{red}{4} \\
&&&&&&$\uparrow$ \\
5 & & \textcolor{red}{6} &$\rightarrow$& \textcolor{green}{7} &$\rightarrow$& \textcolor{red}{8} \\
&&$\uparrow$&&&& \\
\textcolor{red}{9} &$\rightarrow$& \textcolor{red}{10} & & 11 & & 12 \\
$\uparrow$&&&&&& \\
\textcolor{blue}{13} & & 14 & & 15 & & 16 \\
\hline
\end{tabular}
\end{center}
\subparagraph*{\textbf{.2)}}Pour cette règle on fait exactement comme avant, pour trouver entree, sortie etc, la seule chose qui change, c'est qu'il faut modifier la règle \textbf{batterie(Pieces,Batterie,Reste)}, j'ai donc fait la règle \\ \textbf{batterie$\_$tweet(Pieces,Batterie,Reste)}, qui plutôt que les chemins qui ont un reste de énergie positive, elle accepte que les chemins qui donnent de l'énergie positive.
\\ \\ \tab L'appel \textbf{reussite$\_$tweet(15,Pieces)}, on a les résultats suivants:
\\ \\ \tab (X -> symbolise le chemin no réalisable). 

\begin{center}
\begin{tabular}{|ccccccc|}
\hline
\textcolor{blue}{1} & \textbf{X} & \textcolor{red}{2} &$\leftarrow$ & \textcolor{red}{3} &$\leftarrow$& \textcolor{red}{4} \\
&&&&&&$\uparrow$ \\
5 & & \textcolor{red}{6} &$\rightarrow$& \textcolor{green}{7} &$\rightarrow$& \textcolor{red}{8} \\
&&$\uparrow$&&&& \\
\textcolor{red}{9} &$\rightarrow$& \textcolor{red}{10} & & 11 & & 12 \\
$\uparrow$&&&&&& \\
\textcolor{blue}{13} & & 14 & & 15 & & 16 \\
\hline
\end{tabular}
\end{center}

\begin{center}
\begin{tabular}{|ccccccc|}
\hline
\textcolor{blue}{1} &\textbf{X} & 2 &\textbf{X} & \textcolor{red}{3} &$\leftarrow$ & \textcolor{red}{3} \\
&&&&&&$\uparrow$ \\
5 & & \textcolor{red}{6} &$\rightarrow$& \textcolor{green}{7} & $\rightarrow$ & \textcolor{red}{8} \\
&&$\uparrow$&&&& \\
9 & & \textcolor{red}{10} & & 11 &  & 12 \\
&&$\uparrow$&&&& \\
13 & & \textcolor{red}{14} &$\leftarrow$& \textcolor{red}{15} &$\leftarrow$& \textcolor{blue}{16} \\
\hline
\end{tabular}
\end{center}


\end{document}